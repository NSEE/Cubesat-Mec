% ------------------------------------------------------------------------
% Projeto de pesquisa - FAPESP
%
% Modelo criado por: Rodrigo Romano
% "Rodrigo A. Romano" <rromano@maua.br>
%
% Alteraćoes: Rafael Corsi
% "Rafael Corsi" <corsiferrao@gmail.com>
% 
% ------------------------------------------------------------------------
\documentclass[12pt,a4paper]{article}
\usepackage[a4paper,left=3.5cm,right=1.5cm,top=3cm,bottom=2.5cm]{geometry}


\usepackage[utf8]{inputenc}       %Permite acentuação
\usepackage[T1]{fontenc}
\usepackage[english,brazil,brazilian]{babel}
\usepackage[centertags]{amsmath}

\usepackage{lmodern,
			url,
			hyperref,
			multirow,
			array,
			indentfirst,
			ncccomma,
			pst-node,
			pstricks,
			schemabloc,
			amssymb,
			amsfonts,
			graphicx,
			etoolbox, 
			pdfpages,
			hyperref,
			paralist,
			tabularx,
			pdflscape,
			color,				
			doc,
			vhistory}
			
			
\usepackage[titletoc,toc,title]{appendix}

% Reduz espacamento na bibliografia
\usepackage{natbib}
\setlength{\bibsep}{0.0pt}

%Matem figura na secção !
%\usepackage[section]{placeins}

% sub figuras
%ajuste do tamanho entre figuras
\usepackage{subfigure}
\setlength\subfigcapmargin{0.8em}


% Fuzz 
\hfuzz1pt % Don't bother to report over-full boxes if over-edge is < 2pt

% Numeração das notas de rodapé 
\renewcommand{\thefootnote}{\fnsymbol{footnote}}

% Different font in captions 
\newcommand{\captionfonts}{\small}

\makeatletter  % Allow the use of @ in command names
\long\def\@makecaption#1#2{%
  \vskip\abovecaptionskip
  \sbox\@tempboxa{{\captionfonts #1: #2}}%
  \ifdim \wd\@tempboxa >\hsize
    {\captionfonts #1: #2\par}
  \else
    \hbox to\hsize{\hfil\box\@tempboxa\hfil}%
  \fi
  \vskip\belowcaptionskip}
\makeatother   % Cancel the effect of \makeatletter

% Espaçamento
\setlength{\parindent}{30pt} \setlength{\parskip}{6pt}
\newlength{\defbaselineskip}
\setlength{\defbaselineskip}{\baselineskip}

\newcommand{\setlinespacing}[1]{\setlength{\baselineskip}{#1 \defbaselineskip}}
% ---
\newcommand{\PreserveBackslash}[1]{\let\temp=\\#1\let\\=\temp}
\let\PBS=\PreserveBackslash %
\def\baselinestretch{1}
\setlinespacing{1.5}


%Alteraćoes Rafael C 

% Use the microtype package for better typography
% http://www.khirevich.com/latex/microtype/
%
% activate={true,nocompatibility} - activate protrusion and expansion
% final - enable microtype; use "draft" to disable
% tracking=true, kerning=true, spacing=true - activate these techniques
% factor=1100 - add 10% to the protrusion amount (default is 1000)
% stretch=10, shrink=10 - reduce stretchability/shrinkability (default is 20/20)
\usepackage[activate={true,nocompatibility},
			final,
			tracking=true,
			kerning=true,
			spacing=true,
			factor=1100,
			stretch=10,
			shrink=10]
			{microtype} 

% Possibilita que caracteres saiam da margem p/ melhor caber o texto
\SetProtrusion{encoding={*},
			   family={bch},
			   series={*},size={6,7}}
               {1={ ,750},2={ ,500},3={ ,500},4={ ,500},5={ ,500},
               6={ ,500},7={ ,600},8={ ,500},9={ ,500},0={ ,500}}

\microtypecontext{spacing=nonfrench}
               
% Todo notes
\usepackage[disable]{todonotes}
%\usepackage[colorinlistoftodos]{todonotes}  
\setlength{\marginparwidth}{3cm}
\reversemarginpar             
\makeatletter\let\chapter\@undefined\makeatother

\newcommand\todoin[2][]{\todo[inline, caption={2do}, #1, color={red!100!green!33}]{
\begin{minipage}{\textwidth-4pt}#2\end{minipage}}}

% Restora contagem da nota a cada pagina
\usepackage{perpage}
\MakePerPage{footnote}

% Qualidade e compressão do pdf
\pdfminorversion=5
\pdfcompresslevel=9
\pdfobjcompresslevel=2
